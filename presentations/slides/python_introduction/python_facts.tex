\subsection*{Facts about Python}

\begin{frame}
  Timeline
  \begin{itemize}\itemsep=.25cm
    \item{Began in 1989 at CWI as a successor of ``ABC''}
    \item{Initially developed by Guido van Rossum, now Benevolent Dictator for Life}
    \item{Primarily thought of initally as a teaching language}
    \item{Early release cylce every 6 months or so, currently every 2 years}
    \item{Currently core is moving to 3.0, but most libraries only work on 2.7}
  \end{itemize}
\end{frame}



\begin{frame}
  Technical Specifications
  \begin{itemize}\itemsep=.25cm
    \item{Python is a ``multi-paradigm'' language:
      \begin{itemize}
      \item{Imperative}
      \item{Object Oriented}
      \item{Dynamically Typed -- Interpreted}
      \item{Almost Functional}
      \end{itemize}
    }
  \item{This means that programmers can work in a variety of styles, freely
    intermixing constructs from different paradigms}
  \end{itemize}
\end{frame}

\begin{frame}
  Duck Typing
  \begin{itemize}
  \item If it looks like a duck, then it is a duck.
  \item Dynamically evaluates types
  \item Type (and name) errors not caught until runtime
    \begin{itemize}
    \item Unit testing becomes more important
    \item Good IDE can catch many standard errors
    \end{itemize}
  \end{itemize}
\end{frame}

\begin{frame}
  Speed
  \begin{itemize}
  \item Running Python code is usually 10X slower than C
  \item Writing Python code is usually much much much faster than writing C
  \item Reading Python code is usually much much much faster than reading C
  \item Write Python, profile, refine in C
  \item Several projects are working to speed up Python (PyPy, Cython, ...)
  \end{itemize}
\end{frame}

\begin{frame}
  Gotchas
  \begin{itemize}
  \item No multithreading support
  \item Import problem
  \end{itemize}
  But who cares, let's start hacking!
\end{frame}
%% \begin{frame}[fragile=singleslide]
%%   \begin{itemize}
%%   \item \cpp{} is often said to be a ``superset'' of C, but
%%   \end{itemize}
%% \vspace{.15in}
%%   \begin{equation}
%%     \nonumber
%%     \framebox{Superset of C}
%%     \nRightarrow
%%     % \fbox{any C code will compile with a \cpp{} compiler}
%%     \framebox{Any C code will compile with a \cpp{} compiler}
%%   \end{equation}
%%   \begin{itemize}
%%   \item{Simple example: \cpp{} introduces new keywords, so legal C code such as:
%%       \begin{columns}
%%         \column{.3\textwidth}
%%         \begin{block}{}
%% \begin{semiverbatim}
%% int class = 0;
%% int public = 0;
%% int private = 0;
%% int new = 0;
%% \end{semiverbatim}
%%       \end{block}
%% \end{columns}
%% %
%% \vspace{1ex}
%% will not compile with a \cpp{} compiler.
%%     }
%%   \end{itemize}
%% %   \framebox[.3\textwidth]{superset of C}
%% %   $\nRightarrow$
%% %   \framebox[.3\textwidth]{any C code will compile with a \cpp{} compiler}

%% %\begin{columns}[c]
%% %\column{.15\textwidth}
%% %Superset of C
%% %\column{.1\textwidth}
%% %$\nRightarrow$
%% %\column{.45\textwidth}
%% %Any C code will compile with a \cpp{} compiler
%% %\end{columns}
%% \end{frame}



%% \begin{frame}
%%   \begin{itemize}\itemsep=.75cm
%%   \item {Knowledge of C is \emph{absolutely not} required to begin programming in \cpp}
%%   \item {In fact, extensive C knowledge can be detrimental to learning \cpp{} because:
%%       \begin{itemize}
%%       \item{You already ``know'' how to do things in C, }
%%       \item{Therefore you are not exposed to the ``\cpp{} way'' of doing them }
%%       \item{Hence, you are still writing C code, but with a slower, more complicated
%%           compiler and worse binary compatibility!}
%%       \end{itemize}
%%     }
%%   \end{itemize}
%% \end{frame}



% LocalWords:  metaprogramming
