\subsection{Data types}

\begin{frame}
Numpy has a sophisticated view of data.

\begin{block}{}
\begin{tabular}{lll}
bool & int & int8 \\
int16 & int32 & int64 \\
uint8 & uint16 & uint32 \\
uint64 & float & float16 \\
float32 & float64 & complex\\
complex64 & complex128 & \\
\end{tabular}
\end{block}
\end{frame}


\begin{frame}[fragile]
\begin{block}{Using NumPy types}
\begin{minted}{python}
In[55]: np.array([1, 2, 3], dtype='f')
array([ 1.,  2.,  3.], dtype=float32)
In[56]: z.astype(float)
array([  0.,  1.,  2.])
In[57]: np.int8(z)
array([0, 1, 2], dtype=int8)
In[58]: d = np.dtype(int)
In[59]: d
dtype('int32')
In[60]: np.issubdtype(d, int)
True
In[61]: np.issubdtype(d, float)
False
\end{minted}
\end{block}
\end{frame}


\begin{frame}[fragile]
Structured data
\begin{block}{}
\begin{minted}{python}
In[62]: x = np.zeros((2,),dtype=('i4,f4,a10'))
In[63]: x[:] = [(1,2.,'Hello'),(2,3.,"World")]
In[64]: dt = dtype([('time', uint64),
...        ('pos', [
...           ('x', float),
...           ('y', float))])
In[65]: x = np.array([
...        (100, ( 0, 0.5),
...        (200, ( 0, 10.3),
...        (300, (5.5, 15.1)],
...        dtype=dt)
In[66]: x['time']
In[67]: x[ x['time'] >= 200 ]
\end{minted}
\end{block}
\end{frame}
