\subsection{Local Commands}
\begin{frame}[fragile]
\begin{verbatim}
$ ssh lonestar.tacc.utexas.edu
$ module load git
$ git help
usage: git ...
$ git help init
GIT-INIT(1)                        Git Manual                       GIT-INIT(1)
...q
$ git config --global user.name "John Doe"
$ git config --global user.email johndoe@example.com
\end{verbatim}
\end{frame}
\begin{frame}[fragile]
\begin{block}{git init}
Create an empty git repository or reinitialize an existing one
\end{block}
\begin{verbatim}
$ mkdir git_test
$ cd git_test
$ git init
Initialized empty Git repository in
/home1/01392/aterrel/git_test/.git/
\end{verbatim}
\end{frame}

\begin{frame}[fragile]
\begin{block}{git add}
Add file contents to the index
\end{block}
\begin{verbatim}
$ echo "Hello Git World" >> README
$ git add README
\end{verbatim}
\end{frame}

\begin{frame}[fragile]
\begin{block}{git status}
Show the working tree status
\end{block}
\begin{verbatim}
$ git status
# On branch master
#
# Initial commit
#
# Changes to be committed:
#   (use "git rm --cached <file>..." to unstage)
#
#	new file:   README
#
\end{verbatim}
\end{frame}

\begin{frame}[fragile]
\begin{block}{git commit}
Record changes to the repository
\end{block}
\begin{verbatim}
$ git commit -m "Adding README"
[master (root-commit) 774c810] Adding README
 1 file changed, 1 insertion(+)
 create mode 100644 README
\end{verbatim}
\end{frame}

\begin{frame}[fragile]
\begin{block}{git log}
Show the commit logs
\end{block}
\begin{verbatim}
$ git log
commit 774c81087d052e43a630db7f676cfd9a6b006772
Author: Andy R. Terrel <andy.terrel@gmail.com>
Date:   Tue Jul 24 17:53:04 2012 -0500

    Adding README
\end{verbatim}
\end{frame}

\begin{frame}[fragile]
\begin{verbatim}
$ echo "Line 2" >> README
$ git add README
$ git commit -m "Adding Line 2"
$ echo "Line 3" >> README
$ git add README
$ git commit -m "Adding Line 3"
$ echo "Clear file" > README
$ git add README
$ git commit -m "Clear file"
\end{verbatim}
\end{frame}

\begin{frame}[fragile]
\tiny
\begin{verbatim}
$ git log
commit c0513dbf6b609715f1510c438b9d00f065f7f3f4
Author: Andy R. Terrel <andy.terrel@gmail.com>
Date:   Tue Jul 24 17:58:53 2012 -0500

    Clear file

commit 88d4a87be3e7444d06463108e98ca78802f4859e
Author: Andy R. Terrel <andy.terrel@gmail.com>
Date:   Tue Jul 24 17:58:25 2012 -0500

    Adding Line 3

commit 43a446bedd92946d0ccf6fa2218f623284695f8b
Author: Andy R. Terrel <andy.terrel@gmail.com>
Date:   Tue Jul 24 17:58:01 2012 -0500

    Adding Line 2

commit 774c81087d052e43a630db7f676cfd9a6b006772
Author: Andy R. Terrel <andy.terrel@gmail.com>
Date:   Tue Jul 24 17:53:04 2012 -0500

    Adding README
\end{verbatim}
\end{frame}

\begin{frame}[fragile]
\begin{block}{git diff}
Show changes between commits, commit and working tree, etc
\end{block}
\begin{verbatim}
$ git diff 88d4a87be3
diff --git a/README b/README
index d5c15a2..fcb6062 100644
--- a/README
+++ b/README
@@ -1,3 +1 @@
-Hello Git World
-Line 2
-Line 3
+Clear file
\end{verbatim}
\end{frame}

\begin{frame}[fragile]
\begin{block}{git reset}
Reset current HEAD to the specified state
\end{block}
\begin{verbatim}
$ git reset 43a446bedd92
README: needs update

< fix file >
$ git add README
$ git commit -m "Don't clear line this time"
Created commit 0c1757b: Don't clear line this time
 1 files changed, 2 insertions(+), 2 deletions(-)
\end{verbatim}
\end{frame}

\begin{frame}[fragile]
\begin{block}{git checkout}
Checkout a branch or paths to the working tree
\end{block}
\begin{verbatim}
$ git checkout 43a446bedd9294 -- README

$ cat README
Hello Git World
Line 2
$ git add README
$ git commit -m "Going back to line 2"
Created commit e56f835: Going back to line 2
 1 files changed, 2 insertions(+), 2 deletions(-)
\end{verbatim}
\end{frame}

\begin{frame}[fragile]
\begin{block}{git checkout}
Checkout a branch or paths to the working tree
\end{block}
\tiny
\begin{verbatim}
$ git checkout 88d4a87be3e7
Note: checking out '88d4a87be3e7'.

You are in 'detached HEAD' state. You can look around, make experimental
changes and commit them, and you can discard any commits you make in this
state without impacting any branches by performing another checkout.

If you want to create a new branch to retain commits you create, you may
do so (now or later) by using -b with the checkout command again. Example:

  git checkout -b new_branch_name

HEAD is now at 88d4a87... Adding Line 3

$ git checkout master
Previous HEAD position was 88d4a87... Adding Line 3
Switched to branch 'master'
\end{verbatim}
\end{frame}
